%%This is a very basic article template.
%%There is just one section and two subsections.
\documentclass{article}

\usepackage{listings}

\lstset{language=C++, tabsize=4}

\begin{document}

\title{std::bit\_swap and std::byte\_swap}
\author{Klemens D. Morgenstern}
\date{\today}

\maketitle

\newpage
\section*{Abstract}

This proposal introduces two functions to handle differences in endianess. Since
this may be bit- as well as byte-endianess, two functions are proposed
\textit{std::bit\_swap} and \textit{std::byte\_swap}

\section{Problem}

When dealing with lowlevel code, especially in embedded systems, the bit and
byte endianess can become an issue. This is because of inter-device
communication and direct access to the hardware. That is, hardware registers do not have to
be the same as the processor endianess. To solve this issue functions are
necessary.

To use them efficiently with embedded system however, a view constraints have
to be introduced.

\section{Existing Solutions}
Since the functionality are widely needed a few implementations for
byte-swapping are available.

\subsection{qendian.h} 
The Qt framework provides functions to swap the byte-endianess. It does so by a
simple for-loop as of Qt5.

\subsection{Gcc}
The Gnu compiler collection provides a header (byteswap.h) with macros to swap
the bytes 16-, 32- and 64-bit values. They implement the conversion via shifting.
Also the functions \lstinline {__builtin_bswap32} and
\lstinline {__builtin_bswap32} are provided for the same reason. They however
aim to use underlying compiler commands instead of just providing a perprocesser
macro.
\subsection{intrin.h}
The MSVCompiler provides a header with several functions for byte-conversion,
namely \lstinline {_byteswap_ushort}, \lstinline {_byteswap_ulong} and
\lstinline {_byteswap_uint64}. They can be regared as similar to the GCC builtin
functions.

\subsection{Bit-Endianess}
As for now implementations fore Byte-Endianess were shown, so the questions
arises, why bit-endianess should be an issue. It is true, that only very obscure
architecture acutally have a MSB bit-endianess. Though this is true,
an embedded controller can always communicate with a wide variety of devices, which
may include e.g. DSPs with a different structure. Hence it may be necessary to
read data with an inverse bit-order from registers (e.g. a communcation buffer).
There are however no intrinsic functions 

\section{Proposal}

The core proposal is two provide two functions,
\begin{itemize}
  \item \lstinline {std::bit_swap}
  \item \lstinline {std::byte_swap}
\end{itemize}
for each integer type. Though they perform different tasks they are related in
their task. Since there are other proposals, as for example \lstinline
{bit_reference}, we propose to add a header named \lstinline {<bit>}, which
ought to contain all utilities related to bit manipulation.
\subsection{bit\_swap}
The bit\_swap function shall take a reference to an integer type as argument and
return a value with the same bits in the opposite order of the same type.
\begin{lstlisting}
std::uint16_t i = 0b1110001111100001;
auto _i = std::bit_swap(i); 
assert(_i == 0b1000011111000111);
\end{lstlisting}
\subsection{byte\_swap}
Thie byte\_swap function shall take a reference to an integer type as argument and
return a value with the same bytes in the opposite order of the same type. The
bit order in those bytes shall be the same as passed in.
\begin{lstlisting}
std::uint32_t i = 0xDEADBEEF;
auto _i = std::byte_swap(i);
assert(_i == 0xEFBEADDE);
\end{lstlisting}
\subsection{Behaviour}
The functions shall take a const reference to the value passed in.
It shall be possible to call all the functions with a volatile reference. In the
case of a volatile reference it shall be guaranteed, that the operation is
performed with one read from the passed reference. If this is not possible, the
attempt to use a volatile reference shall result in a compiler-error.
\section{Why in the standard?}
For each added standard proposal, it is to answer why it should be a standard
function. The first reason is, that this feature is highly needed on embedded
systems, and thereby need support on many platforms.
Additionally, those function my be implemented in pure C++, but may also make
use of the processor properties.  Therefore this functionality while being very
low-level is highly dependent on the target-architecture and therefore highly
optimizable. This is much more efficiently done by the compiler and it's
stnadard-library than by a library.

\subsection{Architecture Instructions}
For the proposed feature, we can look into machine architectures. The
reason is, that we should assert the availability of the low-level operations on
major architectures. This is not critical but helpful, when deciding if a
standard-lib solution can do more than a manual implementation.
\subsubsection{ARM}
The ARM Architecture provides three instructions which concern this topic:\\

\begin{tabular}{l | l}
  \textit {REV} & Reverse byte order in a word.\\ 
  \textit {REV16} & Reverse byte order in each halfword
  independently.\\
  \textit {RBIT} & Reverse the bit order in a 32-bit word.
\end{tabular}\\

That is we have the necessary functions via assembler, commands. Swapping a
64-Bit Integer (be it by bits or bytes) is not doable in one task for the
architecture. 
\subsubsection{x86}
The x86 Architecture only provide two instructions to swap the bytes:\\

\begin{tabular}{l | l}
  \textit{XCHG}  & Exchange data, i.e. swap bytes for a word.\\
  \textit{BSWAP} & Reverses the byte order of a 32-bit register.
\end{tabular}\\
A bit inverse instruction is not available.
 

\end{document}
