%%This is a very basic article template.
%%There is just one section and two subsections.
\documentclass{article}

\usepackage{listings}
\usepackage{color}
\usepackage[margin=0.25in]{geometry}
\usepackage{relsize}
\lstset{language=C++, tabsize=4}
\usepackage{amsmath}




\begin{document}

\title{Registers for C++}
\author{Klemens D. Morgenstern}
\date{\today}

\maketitle

\tableofcontents
\newpage

 \section*{Abstract}
 
This paper proposes a new language feature, registers. Registers shall replace
bitfields, which should be deprecated. It will thereby provide an easy and
typesafe way to access bits in memory, while keeping the C++ semantics
consistent. It will not introduce new keywords, but a reuse of the old
``register'', which is already deprecated.

\section{Problem}
In current C++, there is not clear way how to access a compound of bits, if
there is no standard type for that. This problem needs a solution, which allows
the programmer to access a sequence of bits which should be modeled as different
values with different types. There is no good solution for this problem, but it
is especially necessary for embedded systems. Very view controllers have a
low-level API which wraps around the hardware in a safe and perfomant matter, so
much of this code has to be written. At the moment there are two used solutions
for this problems.

\subsection{Bitfields}
A popular solution are the C-Bitfields, which are members of compound types:
\begin{lstlisting}
struct bitfield
{
	int x : 14;
	unsigned y : 12;
};
\end{lstlisting}
The problem with this approach is, that the standard doesn't enforce any
aligment, i.e. the compiler may put the bits wherever it pleases. This is on
purpose, since bitfields were introduced to allow the saving of memory. This now
makes the bitfield utterly useless for hardware-access.
The logic of a bitfield is, that a bitfield is a non-addressable member of any
compound type. Therefore there is no given guarantee for any alignment and the
existance of a bitfield has no effect on the struture. It is basicly just a way
to express, that a member is smaller than it's type. 
This is confusing and inconsistent, since no member-pointers can be used for the
bitfields which behave like regular members in other use cases. 
If this proposal gets approved, bitfields used for the implemention of
hardware-register access can easily be migrated to registers. Bitfields used in
other classes, i.e. a mixed style, can and should be changed into regular
members. Hence Bitfiels should be deprecated.
\subsection{Masking \& Shifting}
The more popular approach is to introduce masks and offsets, which are constant
expressions. 
An example would look like this:
\begin{lstlisting}
//old value
old_value = (MASK & some_register) >> OFFSET ;
//write
some_register = (~MASK & some_register) 
			  | (new_value << OFFSET);  
\end{lstlisting}
The example, despite getting no error for using the wrong mask or offset,
works with pure integral-values. Hence no type-safety (as for example
introduced by \lstinline {enum class}) is provided.
This approach is unsafe in several ways and rather hard to read.
It could be tweaked by using more elaborate preprocessor-constructions, but that
would make it even more error-prone.
It is performant, but very hard to debug and write correctly, so that is has to
be considered quite unsafe.
\section{Solution}
The solution is to introduce a new language feature, called register. The
syntax shall look like this:

\begin{lstlisting}
register my_reg
{
	int i : 12;
	int j : 11;
};
\end{lstlisting}
This looks familiar and only the use of  the \lstinline {register} keyword is
new.
\subsection{Basic rules}
A register will alway use an underlying integer-type, \lstinline {int} if not
specified otherwise. It can also be converted to the underlying type by using a
\lstinline {static_cast} et vice versa.
When writing an element of the register, it gives a guarantee, that the
underlying memory is written exactly once, though it maybe read several times.
This is because the register is ment to be access the direct hardware. With
undefined states in between it can yield unexpected behaviour.
For a volatile access of the register, only one read is allowed, be it for
writing or for reading. 
If this cannot be guaranteed for a type (e.g. using a 64-bit int on a 32-Bit
machine) an error shall be provided.
\subsection{Underlying Type}
The underlying type maybe statet explicitly, as for an enumerator. If none is
declared int is used.
\begin{lstlisting}
register my_reg : unsigned long long
{
	int  : 24;
	int x : 12;
};
\end{lstlisting}
Exceeding the size of the underlying type by the element definition shall yield
an error.
%\newpage
\subsection{Member types}
Every type that can be reduced to an integral value shall be a valid type for a
register. Those types can be mixed with no effect for the alignment. 
Additionally, using a bool in a register shall always result in the bool using
one bit. That is the following code is completly valid:

\begin{lstlisting}[label=lst:cplx]
enum class some_enum { x,y,z};

register reg : std::uint16_t
{
	some_enum se : 3;
	unsigned char uc : 2;
	bool b ; //one bit
	int i : 10 ;	
};
\end{lstlisting}

\subsection{Unnamed members}
Similar to C-Bitfields, unnamed members shall be allowed, to implement a
padding.
Additionally, an unnamed member may have a size of zero. The size will pad the
current bit-position up to the next position of a full byte, i.e. a multiple of
8.
\begin{lstlisting}
register reg : std::uin16_t
{
	int i : 2;
	int : 2 ;
	int j : 2;
	int : 0 ;
	int k : 3;
};
\end{lstlisting}

\begin{tabular}{lclllllllclllllll}
\hline
\multicolumn{1}{|l|}{Bit}   & \multicolumn{1}{r|}{0} & \multicolumn{1}{r|}{1} & \multicolumn{1}{r|}{2} & \multicolumn{1}{r|}{3} & \multicolumn{1}{r|}{4} & \multicolumn{1}{r|}{5} & \multicolumn{1}{r|}{6} & \multicolumn{1}{r|}{7} & \multicolumn{1}{r|}{8} & \multicolumn{1}{r|}{9} & \multicolumn{1}{r|}{10} & \multicolumn{1}{r|}{11} & \multicolumn{1}{r|}{12} & \multicolumn{1}{r|}{13} & \multicolumn{1}{r|}{14} & \multicolumn{1}{r|}{15} \\ \hline
\multicolumn{1}{|l|}{Byte}  & \multicolumn{8}{c|}{0}                                                                                                                                                                                & \multicolumn{8}{c|}{1}                                                                                                                                                                                      \\ \hline
\multicolumn{1}{|l|}{Usage} & \multicolumn{2}{c|}{i}                          & \multicolumn{2}{c|}{/}                          & \multicolumn{2}{c|}{j}                          & \multicolumn{2}{c|}{/}                          & \multicolumn{3}{c|}{k}                                                    & \multicolumn{5}{c|}{\textit{/}}                                                                                                 \\ \hline
                            & \multicolumn{1}{l}{}   &                        &                        &                        &                        &                        &                        &                        & \multicolumn{1}{l}{}   &                        &                         &                         &                         &                         &                         &                        
\end{tabular}

\subsection{Signed Interpretation}

The interpretation of signed elements in registers follows the same rules as for
bitfields. If the value is signed it is interpreted as a two's complement.

That is, given a element with 3 bits, which has the value \lstinline {0b101}
will be converted into a \lstinline {unsigned char} of the value \lstinline
{0b11111101}.

\subsection{Class rules}
Unlike enumerators, a register can be interpreted as a form of class. Hence the
proposal is, to apply the same rules as for POD-structs. That is, to enforce
them.
\subsection{Bit-Endianess}
The bit-order of the register shall be according to the endianess of the
target-platform. That is the first entry always starts with the lowest bit of
the underlying type.
\begin{lstlisting}
register r : unsigned char
{
	bool b;
	int : 7;
};
r r1;
r1.b = true;
\end{lstlisting}
This would yield the following:

\begin{itemize}
  \item LSB : \lstinline {00000001}
  \item MSB : \lstinline {10000000}
\end{itemize}
Which of course means, that the value representation in C++ will be 0b10000000.
\subsection{Byte-Endianess}
The byte-endianess shall cause the bytes of the register to be aligned as it
would be expected when using the integer with shift operators.
The following code
\begin{lstlisting}
register r : std::uint16_t
{
	int l : 8;
	int h : 8;
};
r r1;
r1.l = 0xBE;
r1.h = 0xEF;
\end{lstlisting}
would yield
\begin{itemize}
  \item Big-Endian 	  : \lstinline {0xBE, 0xEF}
  \item Little-Endian : \lstinline {0xEF, 0xBE}
\end{itemize} 
Which of course has the C++-representation of 0xBEEF. 
\subsection{Initialization}
The initialization shall be similar to the one of POD types.
Aggregate initialization shall also include the unnamed members. 
\begin{lstlisting}
register reg : std::uint32_t
{
	int i : 12;
	int  : 3; //j
	int k : 5;
	int : 2; //l
}

reg r {42, //reg.i
		1, //reg.l
		3  //reg.k
		2};//reg.l
\end{lstlisting}
This is because an anonym member of a register is exactly this: an element which
cannot be accessed, because it has no name. Hence it exists in all accesses to
the structure which do not need a name. This does however not apply to the empty
space at the end of the register. In the example the register has a size of 32
Bits, but uses 22 of them. The other 10 Bits are not used and are not a unnamed
element. 
If one does not want to initialize the unnamed registers, the explicit struct
initialization of C99 can be used:
\begin{lstlisting}
reg r2 { .i = 42, .k = 3};
\end{lstlisting}
In the latter case, all not-initialized members are in an undefined state.

The aggregate initialization is implicitly constexpr, which allows the check of
the register definitions by simple static assertions, i.e.:
\begin{lstlisting}
static_assert(reg{0,0,0b11111,0} == (0b11111 << 15));  
\end{lstlisting}
\subsection{Member Pointers}
It is quite simply possible to create function pointers for registers. They have
to hold two values, the offset and the length, and can thereby access the
register-element. 
\subsection{Deprecate Bitfields}
Bitfields are a language feature, that is inherited from C. It does not work
well with C++, in Fact is breaks consitency. This is because bitfields are not
addressable, yet can be members in any class.
Since this register proposal brings forth a language feature which
solves the problem bitfields was designed for, bitfields can be deprecated.
Additionally, if introduction register member pointers, all elements of compound
datatypes will have a working member-pointer solution, which is not the case for
bitfields.

\newpage
\section{Why a language feature and not a library?}
This question might be posed, because adding a language feature, as simple as it
might be, always make the language more complex. However, building a library
to implement this sort of thing is rather complicated, since we it should
have no overhead. Implementing these features in the compiler on the other hand
seems doable, since most of the necessary features are already present.

The only realiable C++ implementation is to use plain functions. All
other possibilites in involving classes are not realiable, because we might get
padding. Using plain functions renders them unsafe, since the argument will
always be a integral type. So the type-safety is lost.

Consider the following comparison:

\textbf{Register Solution}
\begin{lstlisting}
register my_reg : std::uint8_t
{
	char : 2;
	char i : 4;
};
my_reg r;
r.i = -2; 
// my_reg.i is now 0b1110 .
char c = i;
//c now has the value 0b11111110 
\end{lstlisting}

\textbf{Function Solution}

\begin{lstlisting}
void set_i(volatile std::uint8_t & reg, char i)
{
	//    mask the register | mask i  | shift to pos
	reg = (reg & ~0b111100) | ((i & 0b1111) << 2);
}
char i get_i(const volatile std::uint8_t & reg)
{
	//       mask the bits     | shift to pos
	char i = (reg & 0b111100) >> 2;
	// now we need to add the sign
	if (i & 0b1000) i |= 0b11110000;
	return i;
};
\end{lstlisting}

This of course can be tweaked by using constexpr-functions to generate the
masks and perform the shift-operations. This however does not change the core
problem: it is a lot of unsafe code. This is especially critical, because there
is no method to get error messages.
Having a register solution, which�s correctness
can be checked by a static\_assert is therefor a huge improvement.
% 
% \newpage
% \section{Changes to the standard}
% The basis is N4296.
% \subsection{Deprecation of Bitfields}
% 
% Move Section 9.6 to the Appendix D.  
% 
% 
% \subsection{Register Specification}
% \input{standard.tex}

\end{document}
