%%This is a very basic article template.
%%There is just one section and two subsections.
\documentclass{scrreprt}

%%\usepackage{listings}
%%\usepackage{color}
%%\usepackage[margin=0.25in]{geometry}
%%\usepackage{relsize}
%%\usepackage{hyperref}
%%\usepackage{backnaur}
%%\usepackage{marginnote} 
 

\usepackage[american]
           {babel}        % needed for iso dates
\usepackage[iso,american]
           {isodate}      % use iso format for dates
\usepackage{listings}     % code listings
\usepackage{longtable}    % auto-breaking tables
\usepackage{ltcaption}    % fix captions for long tables
\usepackage{booktabs}     % fancy tables
\usepackage{relsize}      % provide relative font size changes
\usepackage{underscore}   % remove special status of '_' in ordinary text
\usepackage{verbatim}     % improved verbatim environment
\usepackage{parskip}      % handle non-indented paragraphs "properly"
\usepackage{array}        % new column definitions for tables
\usepackage[normalem]{ulem}
\usepackage{color}        % define colors for strikeouts and underlines
\usepackage{amsmath}      % additional math symbols
\usepackage{mathrsfs}     % mathscr font
\usepackage{microtype}
\usepackage{multicol}
\usepackage{xspace}
\usepackage{fixme}
\usepackage{lmodern}
\usepackage[T1]{fontenc}
\usepackage[pdftex, final]{graphicx}
\usepackage{hyperref} 
\usepackage{memhfixc}     % fix interactions between hyperref and memoir


\lstset{language=C++, tabsize=4}
\usepackage{amsmath}

%\input{layout}
%\input{styles}
%\input{macros} 
%\input{tables}
%

\begin{document}

\title{Registers for C++}
\author{Klemens D. Morgenstern
\\\href{mailto:klemens.morgenstern@mw-sc.de}{klemens.morgenstern@mw-sc.de}}

\thispagestyle{empty}
\begingroup
\def\hd{\begin{tabular}{ll}
          \textbf{Document Number:} & {\larger\docno}             \\
          \textbf{Date:}            & \today                      \\
          \textbf{Revises:}         & \prevdocno                  \\
          \textbf{Reply to:}        & Klemens D. Morgenstern      \\
%                                    & Google Inc                  \\
                                    & klemens.morgenstern@gmx.net
          \end{tabular}
}

\newpage


\tableofcontents

\chapter*{Abstract}
 This paper proposes a new language feature, registers. Registers shall replace
bitfields, which should be deprecated. It will thereby provide an easy and
typesafe way to access bits, while keeping the C++ semantics
consistent. It will not introduce new keywords, but a reuse 
``register'', which is already deprecated (as for C++17).

\chapter{Motivation}
In current C++, there is no standard way to model sequences of bits, if they
are different from the standarized integral type. Especially hardware
registers are often sequences of fundamental types, with length of bits
differing from the fundamental types in c++.
This problem needs a solution, which allows the program to
access such sequence in a fast and type-safe way. 
It should also be remarked, that this is often also required to access a
part of the memory with a fixed address. Hence having no byte-offset is crucial.
Additionally, this kind of problem also occurs when writing low-level
communication, where reducing the bit-size is important.
At the moment there are two used solutions for this problems.
\section{Bitfields}
A popular solution are bitfields, which are members of compound types:
\begin{lstlisting}
struct bitfield
{
	int x : 14;
	unsigned y : 12;
};
\end{lstlisting}
The problem with this approach is, that the standard doesn't enforce any
aligment, i.e. the compiler may put the bits in an undefined order with
undefined padding bits in-between.
This is on purpose, since bitfields were introduced to allow the saving of memory. But this
makes the bitfield useless for hardware-access, if no additional
compiler extension is used (e.g. \lstinline {__attribute__((packed))} for gcc).
A bitfield is a non-addressable member of any compound type. 
Therefore there is no given guarantee for any effect on the alignment with the
existance of a bitfield.
This is confusing and inconsistent, since no member-pointers can be used for
the bitfields which behave like regular members in all other use cases. 
If this proposal gets approved, bitfields used for the implemention of
hardware-register access can easily be migrated to registers. Bitfields used in
other classes, i.e. a mixed style, can and should be changed into regular
members. Therefore bitfiels should be deprecated.
\section{Masking \& Shifting}
The more popular approach is to introduce masks and offsets, which are constant
expressions. 
An example would look like this:
\begin{lstlisting}
//old value
old_value = (MASK & some_register) >> OFFSET ;
//write
some_register = (~MASK & some_register) 
			  | (new_value << OFFSET);  
\end{lstlisting}
The example, despite getting no error for using the wrong mask or offset,
works with pure integral-values. Hence no type-safety (as for example
introduced by scoped enums) is provided.
This approach is unsafe in several ways and rather hard to maintain.
This can be optimized by using the preprocessor or template functions, but the
amount of boiler-plate code is still rather high. It's additionally rather hard
to debug, because no error-message are provided.
\chapter{Solution}
The solution is to introduce a new language feature, called register. The
syntax shall look like this:

\begin{lstlisting}
register my_reg
{
	int i : 12;
	int j : 11;
};
\end{lstlisting}
This looks like the bit-field, while only the use of the \lstinline {register}
keyword is new.
\section{Basic rules}
A register will alway use an underlying integer-type, \lstinline {int} if not
specified otherwise. It can also be converted to the underlying type by using a
\lstinline {static_cast} et vice versa.
When writing an element of the register, it gives a guarantee, that the
underlying integral type is written exactly once, though it maybe read several
times. 
This is because the register is ment to be access the direct hardware. With
undefined states in between it can yield unexpected behaviour.
For a volatile access of the register, only one read access is allowed,
including the write operations.
This definition applies to the way the register masks the integer, so it may
still need more than one operation if the value underneath does. 
That is writing an \lstinline {std::uint64_t} on a 32-bit machine might still take two
write-operations, and would then result in a \lstinline {register} based on
\lstinline {std::uint64_t} needing two write operations. This would however
fulfill the requirement of writing a register the same way as the underlying
integer.
This mechanic is needed, when reading a hardware register, because it
may change at any point in-time. That is, a way to obtain consistent data is
necessary.
\section{Underlying Type}
The underlying type may be statet explicitly, like for enumerators.
If none is declared int is used.
\begin{lstlisting}
register my_reg : unsigned long long
{
	int  : 24;
	int x : 12;
};
\end{lstlisting}
Exceeding the size of the underlying type by the element definition shall yield
an error.\\
A register, as well as a reference or a pointer to a register, may be converted
to the underlying type (or a reference or pointer) to the underlying type via a
\lstinline {static_cast}. There is no implicit conversion, so a register cannot
use an operator of the underlying type.
It may of course have it's own operators.
%\newpage
\section{Member types}
Every type that can be reduced to an integral value shall be a valid type for a
register. Those types can be mixed with no effect for the alignment. 
Additionally, using a bool in a register shall always result in the bool using
one bit. 

\begin{lstlisting}[label=lst:cplx]
enum class some_enum { x,y,z};

register reg : std::uint16_t
{
	some_enum se : 3;
	unsigned char uc : 2;
	bool b ; //one bit
	int i : 10 ;	
};
\end{lstlisting}

\section{Unnamed members}
Similar to C-Bitfields, unnamed members shall be allowed, to implement a
padding.
Additionally, an unnamed member may have a size of zero. The size will pad the
current bit-position up to the next position of a full byte, i.e. a multiple of
8.
\begin{lstlisting}
register reg : std::uin16_t
{
	int i : 2;
	int : 2 ;
	int j : 2;
	int : 0 ;
	int k : 3;
};
\end{lstlisting}

\begin{tabular}{l|cccccccccccccccc|}
Bit   & 0  & 1 & 2 & 3 & 4 & 5 & 6 & 7 & 8 & 9 & 10 & 11 & 12 & 13 & 14 & 15\\\hline 
Byte  & \multicolumn{8}{c|}{0} & \multicolumn{8}{c|}{1} \\\hline                                                                                                           
Usage & \multicolumn{2}{c|}{i} & \multicolumn{2}{c|}{/} & \multicolumn{2}{c|}{j}
& \multicolumn{2}{c|}{/}  & \multicolumn{3}{c|}{k} & \multicolumn{5}{c|}{/} 
\end{tabular}
The state of an unnamed member is always undefined. 
\section{Signed Interpretation}
The interpretation of signed elements in registers follows the same rules as for
bitfields. If the value is signed it is interpreted as a two's complement.

That is, given a element with 3 bits, which has the value \lstinline {0b101}
will be converted into a \lstinline {unsigned char} of the value \lstinline
{0b11111101}.
\section{Class rules}
Unlike enumerators, a register can be interpreted as a form of class.
Hence the proposal is, to apply the same rules as for POD-structs.
That is: a register may not have custom constructors or destructors,
not virtual functions and no custom assign operation. It may however have
member-functions and operators.
\section{Conversions}
By default the register is explicitly convertible (i.e. \lstinline 
{static_cast} ) to the underlying integer et vice versa. Implicit conversions
are not possible.
\section{Bit-Endianess}
The bit-order of the register shall be according to the endianess of the
target-platform. That is the first entry always starts with the lowest
(logical) bit of the underlying type.
\begin{lstlisting}
register r : unsigned char
{
	bool b;
	int : 7;
};
r r1;
r1.b = true;

static_assert(static_cast<unsigned char>(r1) == 0b00000001);
\end{lstlisting}
Which of course means, that the value representation in C++ will be 0b00000001,
though the machine may store this differently.
\section{Byte-Endianess}
The byte-order shall be similar to the bits, that is conform to the C++
representation.
\begin{lstlisting}
register r : std::uint16_t
{
	int l : 8;
	int h : 8;
};
r r1;
r1.l = 0xEF;
r1.h = 0xBE;
\end{lstlisting}
The code above of course has the C++-representation of 0xBEEF, independent of
the actual machine.
\section{Initialization}
The initialization shall be similar to the one of POD types. Since unnamed
members are always in an undefined state, they cannot be intialized.
\begin{lstlisting}
register reg : std::uint32_t
{
	int i : 12;
	int  : 3; //j
	int k : 5;
	int : 2; //l
}
//
reg r {42, //reg.i
		3};//reg.k

\end{lstlisting}
Alternativly, the explicit struct
initialization of C99 can be used:
\begin{lstlisting}
reg r2 { .i = 42, .k = 3};
\end{lstlisting}
The aggregate initialization is implicitly constexpr, which allows the check of
the register definitions by simple static assertions, i.e.:
\begin{lstlisting}
static_assert(reg{0,0,0b11111,0} == reg(0b11111 << 15));  
\end{lstlisting}
As a third way, the 
\section{Member Pointers}
It should not be possible to access the member of a register via
pointers or references. In an later addition this might be implemented as an
library-feature, but this would be another proposal.
\section{Deprecate Bitfields}
Bitfields are a language feature, that is inherited from C. It does not work
well with C++, in fact it breaks consitency. This is because
bitfields can not be pointed to, yet can be members in any class.
Since this docuemtn proposes a language feature which
solves the problem bitfields was designed for, bitfields can be deprecated.
\section{Templated Registers}
There is no obvious problem with templating registers, so it should be allowed.
The same rules as for classes apply, especially concerning template
specialization.
This could be useful when implementing a bit structure, which may be 16 or
32-Bit large with different layouts but the same members.
\chapter{Impact on the standard}
This proposal aims to reuse registers, which would move make this no longer
deprecated. It does also aim to deprecate bit-fields, which would make
[class.bit] deprecated.
Additionally a new type of class-declaration would need to be added.

\chapter{Design Decisions}
\section{Conversions}
Implicit conversions are not part of the proposal, because they would need to
include implicitly constructing a register from the underlying type. This would
introduce an ambigious case, as shown below:

\begin{lstlisting}
register reg : int
{
	int : 3;
	int value : 2;
	int : 1;
};

reg r{2};
\end{lstlisting}

Now, as defined by the POD rules, this would yield an assignment of \lstinline 
{reg.r} to 2. But if we were to allow the construction of reg from int, it could
also that, which would render the expression ambiguous.
Since no implicit construction from an integer can be provided, while handling
the registers as POD types, it is necessary to only allow explicit static casts.
\section{Why a language feature and not a library?}
This question might be posed, because adding a language feature, as simple as it
might be, always makes the language more complex. However, building a library
to implement this sort of thing is rather complicated, since we it should
have no overhead. Implementing these features in the compiler on the other hand
seems doable, since most of the necessary features are already present.

All implementations of this would be implemented by function-calls,
though they may be implemented in a smarter way. So a front-end could look like
this (as proposed by Vicente J. Botet Escriba):
\newpage
\begin{lstlisting}
register my_reg : std::uint8_t
{
	char : 2;
	char i : 4;
	bool b ;
	unsigned int u : 1;
};
my_reg r;

//library solution
bit_tuple<entry<void, 2>, entry<char, 4>, entry<bool>, entry<unsigned int, 1>>
tup;

r.i = 2;
set<0>(tup, 2);

auto x1 = r.b;
auto x2 = get<1>(tup);
\end{lstlisting}

However: the underlying backend would still implemented with functions as shown
below.

\begin{lstlisting}
void set_i(volatile std::uint8_t & reg, char i)
{
	//    mask the register | mask i  | shift to pos
	reg = (reg & ~0b111100) | ((i & 0b1111) << 2);
}
char get_i(const volatile std::uint8_t & reg)
{
	//       mask the bits     | shift to pos
	char i = (reg & 0b111100) >> 2;
	// now we need to add the sign
	if (i & 0b1000) i |= 0b11110000;
	return i;
};
\end{lstlisting}
\newpage

First of all, it should be noted that this is a lot of boiler-plate code. But
obviously, that can be taken care of by a well-constructed library. If we do
however compare the assembly generated, it differs. As an example we use the
\textbf{arm-gcc-none-eabi 4.8.2} with \textbf{-O3}. To be fair, the difference
is not as notable with an x86 target, where sometimes the manual code is
shorter. Since ARM is a RISC machine, we can assume, that each instruction takes
about one CPU-cycle. As comparison we use a C-bitfield, which makes the example
look like this:
\begin{lstlisting}
#include <cstdint>

using namespace std;

struct __attribute__((packed)) my_reg
{
	char : 2;
	volatile signed char i : 4;
	bool b : 1;
	int8_t u : 1;
};

static_assert(sizeof(my_reg) == 1, "");

void set_i(my_reg & r, char i) { r.i = i; }
void set_i(volatile std::uint8_t & reg, char i)
{
	//    mask the register | mask i  | shift to pos
	reg = (reg & ~0b111100) | ((i & 0b1111) << 2);
}

char get_i(const my_reg & r) { return r.i; }
char get_i(const volatile std::uint8_t & reg_in)
{
	auto reg = reg_in; //copy it, so we only have one access.
	//       mask the bits     | shift to pos
	char i = ((reg & 0b111100) >> 2)
			  //add the sign	
      		 | ((reg & 0b100000) ?  0b11110000 : 0);
	return i;
};
\end{lstlisting}
\newpage
The produces assembly:
\begin{lstlisting}
set_i(my_reg&, char):
        ldrb    r3, [r0]        @ zero_extendqisi2
        bfi     r3, r1, #2, #4
        strb    r3, [r0]
        bx      lr
set_i(unsigned char volatile&, char):
        and     r3, r1, #15
        ldrb    r1, [r0]        @ zero_extendqisi2
        and     r1, r1, #195
        orr     r1, r1, r3, lsl #2
        strb    r1, [r0]
        bx      lr
get_i(my_reg const&):
        ldrb    r0, [r0]        @ zero_extendqisi2
        sbfx    r0, r0, #2, #4
        uxtb    r0, r0
        bx      lr
get_i(unsigned char const volatile&):
        ldrb    r0, [r0]        @ zero_extendqisi2
        uxtb    r0, r0
        tst     r0, #32
        ubfx    r0, r0, #2, #4
        ite     ne
        mvnne   r3, #15
        moveq   r3, #0
        orrs    r0, r0, r3
        uxtb    r0, r0
        bx      lr

\end{lstlisting}

The compiler does not find the right
optimization in the set-function. The \lstinline {bfi} does provide the
command to insert a particular set of bits into a value.

For the get function the compiler does not optimize correctly either for the get
function - the \lstinline {sbfx} command is made explicitly to load a signed
value from a register. However, this formulation in C++ does not convey that to
the compiler.

Please note, that a register access is the lowest level of communication on an
embeded program. The unnecessary operations for the get-function can slow down
the performance immensly, which is a strong argument for the new feature.
Especially, since the compiler can do the task through bitfields, but they have
all the problems discussed earlier. A library of course could utilize inline-assembler to get the same result. But
this would mean, that it is actually not a pure C++ library, and thus it
provides a strong case to add this to the standard.

%\input{standard}
 
\end{document}
